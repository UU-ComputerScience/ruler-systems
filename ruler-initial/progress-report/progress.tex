\documentclass[a4paper]{article}

\title{Implementation Strategies for Type Rules\\Progress Report}
\author{Arie Middelkoop}

\usepackage{a4}

\makeatletter
    \def\thebibliography#1{\section*{Publications\@mkboth
      {PUBLICATIONS}{PUBLICATIONS}}\list
      {[\arabic{enumi}]}{\settowidth\labelwidth{[#1]}\leftmargin\labelwidth
	\advance\leftmargin\labelsep
	\usecounter{enumi}}
	\def\newblock{\hskip .11em plus .33em minus .07em}
	\sloppy\clubpenalty4000\widowpenalty4000
	\sfcode`\.=1000\relax}
    \makeatother


\begin{document}

\maketitle

\paragraph{Overview.}

Type rules for programming languages are usually given as a formal
specification.  Since these specifications often include
non-algorithmic formulations, these rules cannot be implemented
straightforwardly. For example, this would require \emph{guessing} of
types or the shape of the inference tree at intermediate steps. 

As a consequence, an implementation of a type checking or inference
algorithm will not resemble the original type rules at all.
Furthermore, given an implementation, it is unclear whether it matches
the type rules, and it might also be overly complicated, inefficient
and not adaptable to future changes.

To remedy these shortcomings, we propose \emph{implementation
  strategies} for type rules. A strategy is a transformation on type
rules which removes the non-algorithmic aspects from the rules. The
resulting type rules are then trivial to implement: either by mapping
them directly onto attribute grammars with Haskell support code, or
using other implementation mechanisms.

\paragraph{Current State.}

To explore the design space, we investigated different topics related
to formal type systems and their algorithmic counterparts:
abstractions for type rules (e.g. syntax extensions, generic
programming, meta programming), type systems for type rules and formal
specifications.
 
Due to the complexity of the topic, we then focused on implementation
mechanisms for type rules. More specifically, we used the Essential
Haskell Compiler as a use case for experimenting with the Utrecht
University Attribute Grammar system. We investigated some syntax
extensions and support for meta programming and debugging. Moreover, a
lot of attention was devoted to efficiency.

More recently, we substantiated the concept of a strategy. We have
informally defined a Damas-Milner strategy, and two extensions based
on it: one for the HML type system (Leijen 09), and one for the FPH
type system (Vytiniotis 08). Both type systems deal with first-class
polymorphism, and provide interesting challenges due to many non-algorithmic
aspects in their type rules.

\paragraph{Outlook.}
Based on our results accomplished so far, we are currently formalizing
the strategies mentioned above, in order to make their semantics precise. Then
we can investigate properties, such as preservation of typing and
applicability. In addition, a tool in which type rules can be
modeled is under development. Strategies, however, are currently just
Haskell functions transforming the AST of the type rules. Our main
focus will be on the design of a language to describe strategies with. 

Furthermore, we are developing attribute grammar-based prototype
implementations of the HML and FPH type systems, which help us in the
formalization of the corresponding strategies.

\nocite{ariemtfp08, hage07generic, dolstra08report}

\bibliography{progress}
\bibliographystyle{plain}

\end{document}
