\documentclass[preprint,natbib]{sigplanconf}

\usepackage{amsmath}
\usepackage{amsfonts}
\usepackage{amssymb}
\usepackage{amsthm}
\usepackage{stmaryrd}
\usepackage{txfonts}
\usepackage{float}
\usepackage{mathpartir}
\usepackage{xcolor}
\usepackage{natbib}

\floatstyle{boxed}
\restylefloat{figure}
\setlength{\fboxsep}{1.4pt}

\newcommand\trans[1]{ \textcolor{blue}{\mathtt{transact} \: [\;} #1 \textcolor{blue}{\;]} }

\begin{document}

\title{Type Inference with Transactions}

\authorinfo{Arie Middelkoop}
           {Universiteit Utrecht}
           {ariem@cs.uu.nl}

\section{Introduction}

  A typed language has a type system that specifies which types are valid for
  an expression. A type checker checks if a given type for an expression is
  valid, whereas a type inferencer finds a permissible type if it exists. In the
  latter case the types are not known beforehand. Therefore, types contain place
  holders, commonly known as {\it type variables}, which are gradually filled in
  during inference. These type variables represent write-once mutable state.
  However, we argue that a more complex representation is necessary.

  \paragraph{Challenges.}
  We first give a list of challenges that an inference algorithm faces, and
  describe what ad-hoc techniques inference algorithms use to deal with these
  challenges. We later show an elegant innovative technique to deal with all of these
  challenges, based on a more complex representation of type variables, and
  transactional techniques to hide this additional complexity.

  \begin{description}
  \item[Absence of principality.]
    First-class polymorphism has become desired property of functional languages
    such as Haskell. The classical formal system in this setting, System F,
    lacks principal types. There may be multiple type derivations for an
    expression resulting in incomparable types. There are variations on System F
    that have principal types that either have more expressive types encoding
    the disjunction of the possibilities~[HML], disallowing all but one of the
    possibilities~[FPH], or a hybrid approach where expressions may have more
    complex types but the types of bindings must be System F types.

    During inference, the absence of principality relates directly to
    more than one possible type for a type variable. In the covariant case by
    discovering that there is more than one rule possible to derive a type for
    this variable. In the contravariant case when conflicting types are
    discovered for a type variable, which either indicates a type error, or
    that the actual type for this variable is more general than the conflicting
    types discovered so far.

    Since only a single type for a variable is allowed, inference algorithms
    for such systems use several type signature propagation techniques to know
    beforehand some of the types that are discovered later, and have complex
    unification procedures that integrate instantiation and generalization
    [EH,HML], defer decisions about variables by recording constrains about them
    in global environments [HML,FPH], and force decisions at certain places
    [FPH]. Polluting the unification procedure is already undesirable, and on
    top of that deals each inference algorithm with these subproblems in their
    own way, using specific and hard to combine infrastructure.

    Moreover, there are type-based analyses (such as usage analysis) that
    benefit from a (partially, i.e. module level) closed-world. For such
    analyses having more than one incomparable type for an expression may even
    be desired for specialization purposes.
  \item[Allowing local type variable assumptions.]
    Some language features allow for stronger assumptions about type variables
    in a certain scope:
    \begin{itemize}
    \item When defining an existential, locally the exact type of the existential
      is known, but hidden globally.
    \item Functional dependencies is a type class extension in which one can express
      that selecting an instance has as side effect that more information about
      some types becomes known. These extra demands may then lead to more
      context reduction, until a fixpoint is reached.
    \item With Generalized Algebraic Data Types, hidden type information can be
      recovered. For example, suppose
      that we can only construct a GADT value with a certain constructor if
      the type of this value is $Expr\;Int$. Furthermore, suppose that we are
      in a context where we encounter such a value, but it has the type
      $Expr\;a$ with $a$ a rigid type. We are not allowed to make any
      assumptions about this $a$. What type it actually has is hidden. Unless
      we match against the constructor again, which proves that in the scope of
      the match the $a$ is actually an $Int$.
    \end{itemize}

    We now focus on two typical techniques used by inference algorithms to deal
    with GADT. One solution is to clone the state of the type
    variables, insert the stronger assumptions about the type variables,
    perform inference on the subexpression, and finally merge the results~[GHC].
    Unfortunately, deferring a portion of the type inference to a more
    appropriate time is hard in this case, because the local memory must be
    reconstructed and merged at the appropriate places. This drawback arises
    when inference algorithms use multi-pass techniques.
    
    Another solution is to use a single memory space and at those places where
    conflicts occur, generate proof obligations that the types are the
    same~[my GADT paper]. Efficiency is a drawback, and it complicates
    instance selection when combining with Functional Dependencies.
  \item[Concurrency.]
    Due to the local nature of type inference, a type inferencer is a suitable
    candidate for parallelism. However, the global mutable state of type
    variables is a problem. Access to these variables must be properly
    synchronized, and the concurrent process properly coordinated such that
    for example resolving of overloading is done only after the required
    type information has become available. There are standard techniques
    to deal with the first case, but not for the latter.
%%  \item[Integrating custom unification algorithms.]
%%    hmm, this is more like a side-effect.
  \end{description}

  The ad-hoc solutions to these challenges influence each other, blowing up the
  implementation complexity. Therefore, it is clear that abstractions are needed
  in order to take the above challenges properly into account. However, we are
  not willing to pay for it in terms of performance: a less ad-hoc solution
  must still run in the same order of magnitude considering time and space.

  We present an elegant and novel approach to deal with all of the above. We
  present the idea in abstract terms first and show concrete examples of this
  idea.
  
  \paragraph{Type inference.}
  We first p
  
  In order to explain our idea without limiting ourselves to particular type
  inference technology, 
  
  we first model type inference.
  
  We consider type inference to be the evaluation of type rules.
  
  
  Inference is evaluating type rules.
  
  
  \paragraph{Transactions.}
  The first.
  
  
  
  
  \begin{description}
    \item[Idea 1: type variables represent a tree.]
      Instead of a single type, the mutable state of a type variables is a
      \emph{rose tree} of types. The types in this tree are related to each
      other by means of a consistency relation between the type contained in
      a node of the tree, and the types contained in its direct children. An
      individual consistency relation may be defined for each node.

      Examples of such consistency relations are: the parent type is more
      general than those of the children, the parent type and those of the
      children are equal, the parent type is totally unrelated to the types
      of the children, or the parent type is the disjunction of the types of
      the children.

      The structure of the tree reflects the idea of a \emph{scope} mentioned in
      the above challenges, and the enforcement of the consistency relation to
      the propagation of type information between scopes. Also, the tree
      represents multiple possible types for a type variable.
    \item[Idea 2: transactional access of type variables.]
      Inference rules are evaluated in the context of a transaction. In this
      context, the mutable state of a type variable is the type contained in a
      single node of the tree associated with the type variable.

      There is one root transaction. A premise of a rule can have a
      transaction-annotation, which causes the evaluation of that premise in
      a nested transaction. There is a unique node for each transaction in the
      tree of a type variable. Inside a transaction, there can be only one type
      for a variable. However, across transactions these types are allowed to
      differ. To what extend is entirely up to the consistency relations of the
      trees in question.

      Transactions can fail, restart, commit or be deferred.
      \begin{itemize}
      \item A transaction fails if the
        evaluation of its premise fails. The effects of the transaction on the
        tree are undone and the subtree removed. This may lead to restarts of
        parent transactions (and their nested transactions).
      \item A transaction restarts when a type that it observed during its
        evaluation changes (due to enforcement of the consistency relations). The
       effects of a restarting transaction are not undone, and it therefore may
        only refine choices it made before.
      \item Premises can be annotated with a fixate-annotation. Transactions
        encountered during the evaluation of such a premise are conceptually
        alive since evaluation entered the premise. The transactions commit
        when evaluation of the premise finishes. A transaction commits by fixing
        its values in the trees. All transactions must commit, except for
        deferred transactions.
      \item A transaction can be deferred until some particular types becomes
        known. A deferred transaction actually represents multiple transactions:
        one for each instantiation of this particular type.
      \end{itemize}
  \end{description}
  
\section{Examples}

  \begin{figure}[htp]
  \begin{mathpar}
    \inferrule*[right=VAR/INST]
      { (x :: \forall \overline{\alpha} . \tau) \in \Gamma }
      { \Gamma \vdash x : [ \overline{\alpha \coloneqq \tau'} ] \: \tau }

    \inferrule*[right=APP/GEN]
      { \Gamma \vdash f : \forall \overline{\alpha} . \tau_a \rightarrow \tau_r \\\\
        \Gamma \vdash a : \tau_a \\
        \overline{\alpha} \not\in \Gamma \\
      }
      { \Gamma \vdash f \: a : \tau_r }
    
    \inferrule*[right=LET]
      { \Gamma \vdash e : \tau_x \\\\
        x :: \forall \overline{\alpha} . \tau_x, \Gamma \vdash b : \tau_b \\\\
        \overline{\alpha} = \mbox{ftv} \: \tau_x - \mbox{ftv} \: \Gamma \\
      }
      { \Gamma \vdash \mbox{let} \: x = e \: \mbox{in} \: b : \tau_b }

    \inferrule*[right=LAM]
      { x :: \tau_x, \Gamma \vdash e : \tau_e }
      { \Gamma \vdash \lambda x . e : \tau_x \rightarrow \tau_e }
  \end{mathpar}
  
  \label{fig:example-impred-ts}
  \caption{Impredicative type system.}
  \end{figure}

  Figure~\ref{fig:example-impred-ts} shows a syntax directed type system for
  first class polymorphism. Figure~\ref{fig:example-impred-trans} is a sound
  but incomplete (because it cannot type $\lambda x \:.\; x \: x$, which
  requires invention of polymorphism) transactional
  inference algorithm for this type system.

  \begin{figure}[htp]
  \begin{mathpar}
    \inferrule*[right=VAR/INST]
      { (x :: \tau_1) \in \Gamma \\\\
        \trans{ \tau_1 \equiv \forall \overline{\alpha} . \tau  \hspace{1em}  \tau_2 \equiv [ \overline{\alpha \coloneqq \tau'} ] \: \tau } \\
      }
      { \Gamma \vdash x :  \tau_2 }

    \inferrule*[right=APP/GEN]
      { \Gamma \vdash f : \tau_{a_g} \rightarrow \tau_r \\\\
        \Gamma \vdash a : \tau_a \\\\
        \trans{ \tau_{a_g} \equiv \forall \overline{\alpha} . \tau_a'  \hspace{1em}  \tau_a \equiv \tau_a'  \hspace{1em}  \overline{\alpha} \not\in \Gamma } \\
      }
      { \Gamma \vdash f \: a : \tau_r }
    
    \inferrule*[right=LET]
      { \Gamma \vdash e : \tau_x \\
        x :: \tau_x', \Gamma \vdash b : \tau_b \\\\
        \trans{ \overline{\alpha} = \mbox{ftv} \: \tau_x - \mbox{ftv} \: \Gamma  \hspace{1em} \tau_x' \equiv \forall \overline{\alpha} . \tau_x } \\
      }
      { \Gamma \vdash \mbox{let} \: x = e \: \mbox{in} \: b : \tau_b }

    \inferrule*[right=LAM]
      { x :: \tau_x, \Gamma \vdash e : \tau_e }
      { \Gamma \vdash \lambda x . e : \tau_x \rightarrow \tau_e }
  \end{mathpar}
  
  \label{fig:example-impred-trans}
  \caption{Transactional inferencer.}
  \end{figure}




\end{document}
